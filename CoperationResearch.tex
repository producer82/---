% -*- coding: utf-8 -*-
\documentclass{oblivoir}
\usepackage[utf8]{inputenc}
\usepackage{kotex}

\title{\textbf{넷마블 분석을 통한 효율적인 경영 방법 및 비젼 고찰}}
\author{\textbf{1학년 9반 15번 박태원}}

\begin{document}
	\maketitle
	\pagebreak
	
	\begin{center}
	\tableofcontents
	\pagebreak
	\end{center}

	\begin{abstract}
	
	\end{abstract}

	\section{서론}
		\subsection{배경{\small (동기)} 및 목적}
			게임 엔터테인먼트 산업은 4차 산업혁명의 역풍에 굴하지 않고 꾸준히 각광받고 있는 산업이다. 이에 국내 시장에서는 2015년 최초로 \textbf{10조원}이상의 매출액을 기록한 이후, `16년에는 11조원, `17년에는 12조원, 가장 최근인 `18년에는 13조원을 기록하며 지속적인 성장세를 보여주고 있다. 
			\footnote{한국컨텐츠진흥원 2018 결산 및 2019 전망 보고서} 
			
			넷마블은 국내에서 오랜 시간 명목을 이어오고 있는 게임 퍼블리셔 중 하나이며, 최근에는 리니지M 연 매출 2000억원을 달성하는 등 세계적인 게임 퍼블리셔로 성장하고 있다.
			이러한 관점에서 게임 산업을 연구하기에 적절한 기업이라 판단하였다.
			
		\subsection{절차}
			
		\subsection{선정 동기서}
	
	\section{선행 조사}
		\subsection{기업 개요}
		
		\subsection{기업 연혁}
		
		\subsection{기업 비전}
		
		\subsection{사업 분야}
	
	\section{본론}
		\subsection{방법}
		
		\subsection{시장 분석}
		
		\subsection{환경 분석}
		
		\subsection{경쟁사 분석}
		
		\subsection{마케팅 분석}
		
		\subsection{재무 분석}
		
		\subsection{주가 분석}
	
	\section{결론}
	
	\section{레퍼런스}
	

\end{document}